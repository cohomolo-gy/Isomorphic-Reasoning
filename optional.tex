\documentclass[tikz]{beamer}

\usepackage{eucal,mathrsfs}
\usepackage{tikz-cd}
\usepackage{amsfonts}
\usepackage{enumerate}
  
\usetheme{Darmstadt}
\colorlet{shadecolor}{gray!15}

\newcommand{\propnumber}{} % initialize
\newtheorem*{prop}{Proposition \propnumber}
\theoremstyle{definition}
\newtheorem{defn}{Definition}[section]
\newenvironment{propc}[1]
  {\renewcommand{\propnumber}{#1}%
   \begin{shaded}\begin{prop}}
  {\end{prop}\end{shaded}}

\newcommand{\cat}[1]{\mathbf{#1}}
\newcommand{\homf}[2]{[\cat{#1}, \cat{#2}]}


\title{Isomorphic Reasoning: Counting Polymorphic Type Inhabitants}
\author{Emily Pillmore, Alexander Konovalov}
\date{May 2019}

\begin{document}

\section{Optional}
\subsection{}

\begin{frame}{Adjunctions}
    This section is optional, but a wonderful insight into the nature of duality and the previously discussed theory. 
\end{frame}

\begin{frame}{Adjunctions}
    \textit{"Adjoint Functors arise everywhere" - Saunders MacLane} 
\end{frame}

\begin{frame}{Adjunctions}
    In fact, they are more than pertinent to the study of functional programming, as it is a classic theorem of adjoint functors (also called adjunctions) that every adjunction gives rise to a monad/comonad pair. 
\end{frame}

\begin{frame}{Adjunctions}
    Lets begin with some definitions. 
\end{frame}

\begin{frame}{Adjunctions}
   
   \begin{definition}[Adjunction]
        Let $L: \cat{C} \to \cat{D}$ and $R : \cat{D} \to \cat{C}$ be functors. $L$ and $R$ are called \textit{adjoint} functors if
        there exists a natural isomorphism $\Phi : \cat{D}(L-, =) \cong \cat{C}(-, R=)$. 
   \end{definition}{} 
\\  
The components of $\Phi$ are those morphisms 

\begin{equation*}
    \Phi_{x,y}: \cat{D}(Lx, y) \cong \cat{C}(x, Ry).
\end{equation*}
\end{frame}{}

\begin{frame}[fragile]
   
   \begin{definition}[Unit/Co-unit of an Adjunction]
    Two functors $L: \cat{C} \to \cat{D}$ and $R : \cat{D} \to \cat{C}$ are adjoint if there exist natural transformations $\eta : 1_C \Rightarrow RL $ and $\epsilon : LR \Rightarrow 1_D$ which satisfy the following triangle identity: 
    
    \begin{center}
        $1_{Lx}:$ \begin{tikzcd}
            Lx \ar[r, "L\eta_x"] & LRLx \ar[r, "\epsilon_{Lx}"] & Lx
        \end{tikzcd}
    \end{center}
    
    and 
    \begin{center}
        $1_{Ry}:$ \begin{tikzcd}
                Ry \ar[r, "\eta_{Ry}"] & RLRy \ar[r, "R\epsilon_y"] & Ry
            \end{tikzcd}
    \end{center}
   \end{definition}

\end{frame}

\begin{frame}[fragile]

Let $U: \cat{D} \to \cat{C}$ be a forgetful functor, and let $c \in \cat{C}$. A \textbf{free \cat{D}-object} on $c$ with respect to $U$
is an object of $\cat{D}$ satisfying the universal property that $F$ would have if $F$ would have if it were left-adjoint to $U$. More precisely: a free $\cat{D}$-object on c consists of an object $c \in C$ together with a morphism $f:c \to Ud$ in $C$ such that for any other $d' \in D$ and morphism $g:c \to Ud'$, there exists a unique $h:d \to d'$ in $\cat{D}$ such that $Uh \circ f = g$. Diagrammatically: 

\begin{center}
    \begin{tikzcd}
        c 
            \ar[r, "f"]
            \ar[dr, "g"]
        & Ud \ar[d, "Uh", dashed]
        \\ & Ud'
    \end{tikzcd}
\end{center}
\end{frame}

\begin{frame}{Examples}

Let $U: \cat{Mon} \to \cat{Set}$ be a forgetful functor, nad let $MS$ be the free object in $\cat{Mon}$ generated by set $S$. Any set function from $S$ to the underlying set $UN$ of another monoid $N \in \cat{Mon}$ extends to a unique monoid homomorphism $MS \to N$ per the free construction described in the previous slide. One can check that this construction forms an adjunction $ M \dashv U$ - that is, the following hom-sets are in bijection:

\begin{center}
    \begin{equation*}
        \cat{Mon}(MS, N) \cong \cat{Set}(S, UN)
    \end{equation*}{}
\end{center}

\end{frame}

\begin{frame}[fragile]

Let $J: \cat{J} \to \cat{C}$ be a diagram of shape $\cat{J}$ in $\cat{C}$. A \textbf{cone} in $\cat{C}$ is an object $N \in \cat{C}$ together with a family $\beta_X: N \to JX$ such that for every morphism $Jj: JX \to JY$ we have $\beta_Y = Jj \circ \beta_X$. A \textbf{limit} is a terminal cone $(L, \phi)$ inducing a unique $f: N \to L$ such that all $\psi_{(-)}$ factor through $(L, \phi)$. I.e. $\beta_X = \phi_X \circ f$. Diagrammatically, 

\begin{center}
    \begin{tikzcd}
        & N \ar[d, "f", dashed, description] 
            \ar[dl, "\beta_X", bend left=40]
            \ar[dr, "\beta_Y", bend right=40] \\
        & L \ar
            \ar[dl, "\phi_X", bend left]
            \ar[dr, "\phi_Y", bend right] \\  
        JX \ar[r, "Jj"]
        & & JY
    \end{tikzcd}{}
\end{center}{}
    
\end{frame}

\end{document}