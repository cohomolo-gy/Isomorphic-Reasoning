\documentclass[tikz]{beamer}

\usepackage{hyperref}
\usepackage{enumerate}
\usepackage{mathrsfs}
\usepackage{tikz-cd}

  
\usetheme{Darmstadt}
\colorlet{shadecolor}{gray!15}

\newtheorem{prop}{Proposition}

\theoremstyle{definition}
\newtheorem{defn}{Definition}[section]

\newcommand\dinat{\overset{\cdot\cdot}\to}
\newcommand{\cat}[1]{\mathbf{#1}}
\newcommand{\homf}[2]{[\cat{#1}, \cat{#2}]}


\title{Isomorphic Reasoning: Counting Polymorphic Type Inhabitants}
\author{Emily Pillmore, Alexander Konovalov}
\date{May 2019}

\begin{document}

\section{Optional}
\subsection{}

\begin{frame}{Adjunctions}
    This section is optional, but a wonderful insight into the nature of duality and the previously discussed theory. 
\end{frame}

\begin{frame}[fragile]{Goals}

Here's a sneak peek: 

\begin{prop}[LAPC, RAPL]
    Any category $\mathbf{C}$ admits all limits and colimits indexed by a small category $\mathbf{J}$ if and only if the constant diagram functor $\Delta : \mathbf{C} \rightarrow \mathbf{C}^{\mathbf{J}}$ has a right and left adjoint \textit{colim} $\dashv \Delta \dashv$ \textit{lim}:
   
 \end{prop}
\begin{equation*}
\begin{tikzcd}
    \mathbf{C} 
        \arrow[r, "\Delta" description, line sep] &
    \mathbf{C^{J}}
        \arrow[l, bend left=35, "colim"{name=colim, below}]
        \arrow[l, bend right=35, "lim"{name="lim", above}]
\end{tikzcd}
\end{equation*}

When these functors exist, we can prove our arithmetic!

\end{frame}

\begin{frame}{Adjunctions}
    \textit{"Adjoint Functors arise everywhere" - Saunders MacLane} 
\end{frame}

\begin{frame}{Adjunctions}
    In fact, they are more than pertinent to the study of functional programming, as it is a classic theorem of adjoint functors (also called adjunctions) that every adjunction gives rise to a monad/comonad pair. 
\end{frame}

\begin{frame}{Adjunctions}
    Lets begin with some definitions. 
\end{frame}

\begin{frame}{Adjunctions}
   
   \begin{definition}[Adjunction]
        Let $L: \cat{C} \to \cat{D}$ and $R : \cat{D} \to \cat{C}$ be functors. $L$ and $R$ are called \textit{adjoint} functors if
        there exists a natural isomorphism $\Phi : \cat{D}(L-, =) \cong \cat{C}(-, R=)$. 
   \end{definition}{} 
\\  
The components of $\Phi$ are those morphisms 

\begin{equation*}
    \Phi_{x,y}: \cat{D}(Lx, y) \cong \cat{C}(x, Ry).
\end{equation*}
\end{frame}{}

\begin{frame}[fragile]
   
   \begin{definition}[Unit/Co-unit of an Adjunction]
    Two functors $L: \cat{C} \to \cat{D}$ and $R : \cat{D} \to \cat{C}$ are adjoint if there exist natural transformations $\eta : 1_C \Rightarrow RL $ and $\epsilon : LR \Rightarrow 1_D$ which satisfy the following triangle identity: 
    
    \begin{center}
        $1_{Lx}:$ \begin{tikzcd}
            Lx \ar[r, "L\eta_x"] & LRLx \ar[r, "\epsilon_{Lx}"] & Lx
        \end{tikzcd}
    \end{center}
    
    and 
    \begin{center}
        $1_{Ry}:$ \begin{tikzcd}
                Ry \ar[r, "\eta_{Ry}"] & RLRy \ar[r, "R\epsilon_y"] & Ry
            \end{tikzcd}
    \end{center}
   \end{definition}

\end{frame}

\begin{frame}[fragile]

Let $U: \cat{D} \to \cat{C}$ be a forgetful functor, and let $c \in \cat{C}$. A \textbf{free \cat{D}-object} on $c$ with respect to $U$
is an object of $\cat{D}$ satisfying the universal property that $F$ would have if $F$ would have if it were left-adjoint to $U$. More precisely: a free $\cat{D}$-object on c consists of an object $c \in C$ together with a morphism $f:c \to Ud$ in $C$ such that for any other $d' \in D$ and morphism $g:c \to Ud'$, there exists a unique $h:d \to d'$ in $\cat{D}$ such that $Uh \circ f = g$. Diagrammatically: 

\begin{center}
    \begin{tikzcd}
        c 
            \ar[r, "f"]
            \ar[dr, "g"]
        & Ud \ar[d, "Uh", dashed]
        \\ & Ud'
    \end{tikzcd}
\end{center}
\end{frame}

\begin{frame}{Examples}

Let $U: \cat{Mon} \to \cat{Set}$ be a forgetful functor, nad let $MS$ be the free object in $\cat{Mon}$ generated by set $S$. Any set function from $S$ to the underlying set $UN$ of another monoid $N \in \cat{Mon}$ extends to a unique monoid homomorphism $MS \to N$ per the free construction described in the previous slide. One can check that this construction forms an adjunction $ M \dashv U$ - that is, the following hom-sets are in bijection:

\begin{center}
    \begin{equation*}
        \cat{Mon}(MS, N) \cong \cat{Set}(S, UN)
    \end{equation*}{}
\end{center}

\end{frame}

\begin{frame}[fragile]

Let $J: \cat{J} \to \cat{C}$ be a diagram of shape $\cat{J}$ in $\cat{C}$. A \textbf{cone} in $\cat{C}$ is an object $N \in \cat{C}$ together with a family $\beta_X: N \to JX$ such that for every morphism $Jj: JX \to JY$ we have $\beta_Y = Jj \circ \beta_X$. A \textbf{limit} is a terminal cone $(L, \phi)$ inducing a unique $f: N \to L$ such that all $\psi_{(-)}$ factor through $(L, \phi)$. I.e. $\beta_X = \phi_X \circ f$. Diagrammatically, 

\begin{center}
\begin{tikzcd}[row sep = large]
& N \arrow[d, "f", dashed, description] \arrow[ddr, bend left,"\beta_Y"]   \arrow[ddl, bend right,"\beta_X",swap] & \\
& L \arrow[dr, "\phi_Y"]  \arrow[dl,swap, "\phi_X"] & \\
JX  \arrow[rr,swap, "Jj"] & & JY
\end{tikzcd}
\end{center}
    
\end{frame}

\begin{frame}{Limits and Colimits}
A \textbf{colimit} is defined dually, in a  similar way, with the arrows reversed. 
\end{frame}

\begin{frame}{Limits and Colimits}
Notice how familiar those diagrams look?
\end{frame}

\begin{frame}{Limits and Colimits}
This is because products and coproducts as previously defined are limits and colimits themselves!
\end{frame}

\begin{frame}{Limits and Colimits}
This is because products and coproducts as previously defined are limits and colimits themselves! In fact, limits and colimits define functors $\cat{C}^{\cat{J}} \to \cat{C}$. 
\end{frame}


\begin{frame}{Limits and Colimits}
Define the diagonal functor$ \Delta : \cat{C} \to \cat{C}^{\cat{J}}$ taking objects $c \in \cat{C}$ to the constant $\cat{J}$-shaped diagram. If $\cat{C}$ is a locally small category, then $\cat{C}$ preserves all limits (colimits, resp.) if and only if $\Delta$ admits a right adjoint (left adjoint, resp.). 
\end{frame}

\begin{frame}{Examples}

Let $\cat{C}$ be $\cat{Set}$ or its subcategory $\cat{Fin}$. Note that there exists an adjunction $A \times - \dashv (-)^A$ for any $A$. For any sets $A, B, C, D$, the following natural isomorphisms hold: 

\begin{center}
\begin{equation*}
A \times (B + C) \cong (A \times B) + (A \times C) \qquad (B \times C)^A \cong B^A \times C^A
\end{equation*}

\begin{equation*}
A^{B + C} \cong A^B \times A^C
\end{equation*}
\end{center}


\end{frame}

\begin{frame}{Limits and Colimits}
\begin{proof}
The left adjoint $A \times -$ preserves the coproduct $B + C$, the right adjoint $(-)^A$ preserves the product $B \times C$, and the functor $A^{(-)}: \cat{Set}^{op} \to \cat{Set}$ is mutually right-adjoint to itself, and so carries coproducts in $\cat{Set}$ to products in $\cat{Set}$. The laws of cardinal arithmetic follow by applying the cardinality functor $|-|: \cat{Fin_{Iso}} \to \cat{Fin}$

\end{proof}
\end{frame}


\begin{frame}{Bonus}

Bonus: Functors of the form $\cat{D}^{op} \times \cat{C} \to \cat{Set}$ are known as \textit{profunctors}. Profunctors define proof-relevant relations between objects in (possibly) heterogeneous categories and the sets of morphisms between objects in $\cat{C}^{op} \times \cat{D}$ are called \textit{heteromorphisms}. Adjunctions are natural isomorphisms of profunctors which establishes a correspondence between heteromorphisms in an opposite category, and heteromorphisms in a standard category. Taking this perspective that profunctors denote relations between objects imples that adjunctions encode an isomorphism between dual relations! Very cool. Thanks. 
\end{frame}


\end{document}